\chapter*{Summary}\label{sec:summary}
\addcontentsline{toc}{chapter}{Summary}
\markboth{SUMMARY}{}

Statewide and megaregion travel demand models are used to help formulate plans and policies, evaluate and prioritize projects and programs, and assess the economic and social impacts of major transportation investments. They are used in 39 states today, compared to the 19 states documented as using them in a similar synthesis a decade ago. The broader adoption of these models reflects the need to evaluate policies and investments focused on planning for megaregions, intercity travel and freight planning, and better data for statewide planning. In addition to their wider use they have increasingly adopted advanced modeling techniques pioneered in urban areas, such as microsimulation of synthetic households and firms, activity-based demand models, and dynamic traffic assignments. These approaches are applied to both person and freight travel analyses and are poised to take advantages of emerging data and techniques.

The goal of this synthesis is to document current practices and emerging trends in statewide and megaregion travel forecasting. Several specific research questions were posed to focus the investigation into issues such as the motivation for and capabilities of statewide and megaregion models, options for extending them and challenges states face doing so, data requirements, how they are used and integrated with other models, institutional constraints, and emerging trends and methods. Several related issues were investigated as well, to include performance monitoring, alternative forecasting approaches, and how states are addressing hot topics such as pricing and congestion, multi-state corridors, and new transportation modes. 

Several means were used to gather and assimilate the information required to answer these questions. The primary approach consisted of a survey of voting members of the AASHTO Standing Committee on Planning, who were asked to coordinate with modelers in their state to complete an online survey about their practices and issues. Forty-six DOTs responded to the survey (an 92 percent response rate). Telephone interviews were used to gather additional information in some cases and clarify survey responses. Further insights were obtained through an informal survey of members of the Transportation Research Board's Subcommittee on Statewide Modeling. Its members are the most prominent developers and users of such models, about a dozen of whom provided insights that transcended single states or modeling approaches.

There are no operational megaregion models in the USA, aside from high-speed rail (HSR) models used for specific projects. Researchers and FHWA planners active in megaregions were interviewed to gain insight into issues and opportunities in this area. Information about national travel models was obtained from FHWA sponsors of such research, as well as draft reports they shared about such efforts. Finally, a literature review was also conducted, as well as reviews of model documentation, where available. 

Significant progress has been made in statewide modeling over the past decade. Several common themes emerged from the review of the information gathered in this study:
\begin{itemize}
\item There is much to celebrate in the results. Statewide models have flourished, and have significantly expanded detail and capabilities compared to those just a decade ago. These advances occurred despite lingering and significant gaps in data and understanding of several long-distance person and visitor travel and how it is changing over time.
\item There is a wide diversity of demands placed upon such models. They range from high-level planning analyses such as statewide transportation plans and project prioritization to detailed analyses of specific projects in rural areas, with most encompassing freight as well as person travel. There is no such thing as a typical or best practice, for best must be viewed in light of the unique blend of analytical requirements, capabilities, and available resources in each state.
\item Understanding freight and its economic linkages and impacts was cited as the primary motivation for statewide models in some cases, and was near the top of the list for those where it was not. Freight analytics are also important in urban models, but to nowhere near the extent they are at the statewide level. Robust models of freight require macroeconomic forecasts that are more detailed than the simple employment estimates that usually suffice for urban areas, dictating the need for parallel investments in and partnerships with other state agencies responsible for economic forecasting. 
\item The importance of investing in data collection cannot be overstated. Many existing models have been built with borrowed, synthetic, or limited National Household Travel Survey (NHTS) data, owing to the lack of resources available for conducting statewide household and establishment travel surveys. The lack of local data has limited the range of issues such models can address and the accuracy at which they can do so. The few states that have made significant investments in data collection -- California, Michigan, Ohio, and Oregon -- have significantly improved their understanding of travel patterns across the state. The metropolitan planning organizations in those states have benefited as well from access to the data.
\item The considerably more sophisticated mode choice models used for HSR forecasting, as well as the data used to develop them, are finding their way into statewide models. California is an excellent example of this synergy, where travel survey data collected by Caltrans were used in the HSR model, and the mode choice model from the latter used in the statewide model. 
\item The transition path for improved person travel models is well established, whether the goal is to improve existing models or move to advanced travel models. Several states, such as Maryland and Texas, have developed multi-year roadmaps for incremental model enhancements. There is no gradual transition for freight modeling, however. A much larger investment in data and model development are required to create activity-based freight models, which limits the number of states that can afford them.
\item Most states appear eager to leverage big data for extending their models. About a dozen states are using truck GPS tracking data from the American Trucking Research Institute to build better truck models, and a similar number have purchased origin-destination matrices of person travel built from mobile device tracking data offered by several vendors. The quality and coverage of these data are increasing as more vendors enter this market, making it likely that big data will play an increasingly prominent role in statewide modeling.
\end{itemize}

Several challenges remain in spite this progress, or in some cases because of it. Statewide modelers have proven creative at borrowing and synthesizing data and ideas from elsewhere, and adept at implementing many of the advances from urban travel modeling. However, this ingenuity is not sufficient to overcome some of the larger obstacles that remain. Some of the more pervasive issues include:
\begin{itemize}
\item The advances in statewide modeling have not been accompanied by similar strides in megaregion modeling. The only known implementation of a purposefully designed megaregion model was a recently completed FHWA demonstration project in the Chesapeake Bay region. A few other ad hoc multi-state models have appeared over the past two decades but were abandoned after their intended use. Ironically, several high-speed rail (HSR) models have market catchment areas that correspond to or connect megaregions. This includes models connecting the LA Basin and San Francisco Bay Area, Dallas-Fort Worth and Houston, and Chicago and St. Louis. However, like earlier ad hoc models they have been commissioned for a specific project and do not appear to have a life beyond it.
\item Attracting and retaining well-qualified modeling specialists is particularly challenging for state departments of transportation, many of whom have much lower salaries and fewer promotion possibilities for non-engineers than metropolitan planning organizations and consulting firms. Such specialists are in high demand, and the inability of states to compete for them reduces what they can accomplish in many cases, and increases their reliance on consultants. Some states, such as Ohio and Oregon, have overcome this problem. However, it remains an issue in other places, making this a barrier for some agencies wishing to expand their statewide modeling capabilities.
\item Most states use the FHWA Freight Analysis Framework (FAF) or similar third-party data to portray long-distance freight flows, which often have trip ends well beyond the state borders. Thus, considerable duplicative effort is spent designing and implementing ways to allocate the FAF data to sub-regional levels and to external gateways or regions within the model. Some states question the accuracy of the data and the small sample sizes of the data fused to create the FAF, despite their dependence upon it.
\item There are no good examples of states pooling funds and talent to undertake large-scale surveys or other expensive enhancements.
\end{itemize}

Statewide models are increasingly being used to examine many of the same complex issues facing urban planners. Evaluating traveler responses to congestion and pricing (to include tolling), equity, network and travel time reliability typically requires modeling at high levels of behavioral, spatial, and temporal resolution. The distinction between urban and statewide models blurs as their levels of resolution converge. Few states appear ready or able to model the entire state at the urban level, despite pressures for increasingly fine-grained analyses. Moreover, many states are starting to grapple with emerging issues such as autonomous vehicles, new mobility services, and the impact of deteriorating infrastructure, all of which tilt them towards more advanced data and models.

These trends will continue to shape the practice of statewide modeling over the next decade. The investments made by states in the data, models and staff have changed the culture in many state agencies. Advances in statewide modeling in Oregon have trickled down to urban models, and increased collaboration with MPO modelers and other state agencies. The benefits of the major investments in data by California, Michigan, Ohio, and Oregon have been shared by their planning partners and have fostered the development of standard urban models and data sharing. Florida and Texas are also notable in this regard. The insights gained in this synthesis of practice suggest these trends will accelerate in the coming years, prompting continued innovation and adoption of statewide models. 
