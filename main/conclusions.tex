\chapter{Conclusions}\label{sec:conclusions}

A small group of modelers met in Florida in late 2004 to share experiences about a dozen statewide models \citep{giaimo05}. While they did not review every statewide model that existed then, those discussed were the most actively used. Today most states have a statewide model, and several more are in development. A significant amount of applied research has been carried out, a vibrant TRB subcommittee devoted to the topic has emerged, and a great deal of data have become available that reduce the burden of building and using such models. A definitive practice has yet to evolve, but this more reflects the diversity of uses and scale at which such models are applied, rather than lack of scientific consensus or broad experience with them. In a sense, the diversity of these models has been their strength, as it has allowed different approaches to be tested in practice, which in turn has guided the development of new data, methods, and platforms for statewide modeling.

The diversity of approaches described in this report also poses challenges, particularly for states uncertain about next steps they should take, or the costs and benefits associated with doing so. Data on long-distance travel remain almost non-existent, precluding the ability to model such flows with confidence or understand their impact upon state and local transportation systems. This is especially so with freight and commercial vehicle data. Thus, progress with such models at all levels has lagged that of person travel demand modeling and forecasting. There is a significant gap between simple freight models based upon borrowed or synthetic data and more sophisticated models based upon robust local data. A major investment is required to bridge the gap, even with the greatly increased availability of federal commodity flow data and experiences with advanced freight models elsewhere. This represents a significant obstacle for many states, for understanding and modeling freight is very important to most agencies, and the top motivation for statewide modeling for some. The focus on freight seems likely to only increase further under the recently passed FAST Act.

The challenges are even larger when considering megaregion models. They share the same technical challenges faced by statewide models, as well as heavy computational and data burdens. Most megaregions encompass several metropolitan areas, each with travel models that have long run times. Part of the challenge of megaregional modeling is reconciling the data and making tradeoffs between model features and tractability. There are also financial challenges, for data collection across such a large area and in diverse markets can be costly, and in some cases, prohibitively so. Finally, there are unique political challenges associated with such models. There is often no agency with the mandate or resources required to apply and maintain such a model, or integrate it into planning within the megaregion.

It is encouraging to note that considerable progress has been made with statewide models despite these limitations. A wide variety of uses were reported, with most used to evaluate travel between or outside of urban areas, long-distance travel, freight, and to better represent visitor and external travel in urban models. Many early statewide models were developed to address one or two issues. They have scaled to address a wider range of them, and emerging methods and data will enhance both the range of issues that can be evaluated with such models, as well as the levels of resolution they can do so at.

How statewide models advance over the next decade will be shaped by decisions made outside of the states. Big data has the potential to radically change how models are built and used. How these data are shaped by vendors that assemble and resell them, at what levels of resolution, and at what price point are unsettled at this writing.

Perhaps even more influential will be decisions about investments in national person and freight travel data and models made by the federal government. A national model of person long-distance travel has been designed for FHWA, but its future is uncertain at this writing. A comparable behavioral freight model is still in the exploratory research phase, and further away from development. The availability of either of these models would significantly improve the data available to statewide modelers, and if designed to accommodate such, policy-sensitive models that significantly augment, if not replace altogether, such models that each state currently develops separately. How states, working singly or together, can develop interim solutions that anticipate the capabilities of national models, remains an unexplored but fertile avenue for development.

The lack of parallel progress in megaregional models is disappointing on several levels. Only one prototype has been completed, now dormant, and none appear on the horizon. Several ad hoc multi-state models have been developed over the past decade that mimic megaregion models, such as the five-state model used to evaluate interstate passenger rail corridors in Texas and surrounding states, and on-going fusion of five statewide models to study the Appalachian Highway System. However, they were built for specific projects, with no expectation of further use. 

There also
appear to be many missed opportunities for states to collaborate on data collection, especially for highly expensive statewide household and establishment travel surveys. While arguably not as compelling for the larger states, there are several regions where smaller states could cooperate to obtain much better data than any of them could afford on their own. A similar case might be made for multi-state models, especially when their travel markets are tightly coupled, or where metropolitan areas near their shared borders create market areas that transcend state boundaries. It is not clear how progress in these areas might be motivated, or how funds can be pooled to make the value proposition appealing to the states involved.

As formidable as some of the data and methodological challenges are, they pale in comparison to the serious shortage of well-qualified staff. Granted, this is a pervasive problem affecting almost all public agencies. However, the more rigid pay scales and requirements for engineering licensure to advance in most state departments of transportation makes the necessary talent much harder to recruit and retain. This is perhaps an even bigger issue in predominately rural states, where statewide modelers also develop travel forecasts for smaller urban areas lacking such staff. Some states have been able to keep a small number of highly qualified modelers, but often the strongest modelers have sought careers outside of state agencies.

Many states have increased their reliance on consultants to fill the in-house staffing gap, with some even preferring outsourcing due to shrinking budgets and positions. In the short term this arguably benefits the agency, for they gain instant access to expertise that would take years to develop themselves, and can quickly match staffing to current requirements. However, such practices are counter-productive in the long term, for much of the expertise and staff bandwidth required to use and maintain the models are lost when the contract ends or lapses. Even the best data and models cannot overcome the loss or lack of staff with the knowledge and experience required to creatively and competently apply them. To the extent that this problem lingers, it represents as big an impediment to further progress with statewide models as the other factors cited above.

None of these issues are insurmountable, and some perhaps easily tackled. It is useful to consider the challenges faced a decade ago, and the progress made since. \cite{miller04} wrote a particularly cogent critique of intercity travel models around the same time the practitioners mentioned above were meeting in Florida. His review focused upon several intercity rail forecasts under review at that time, but the weaknesses he exposed applied equally well to statewide models. 

Miller's call for improved data, increased model disaggregation, deeper consideration of access and egress modes for intercity travel, and advancements in model structure and specification are all better addressed in most statewide models today, and much more so in those used to evaluate proposed HSR systems. Some models have been rebuilt from scratch, as he suggested, while others have improved significantly in their ability to meet the information needs of their sponsors. With big data poised to finally fill long-standing data gaps and the ascendency of autonomous vehicles and mobility as a service to refocus us, it seems all but assured that the next review of statewide models will document even larger strides than those described here.
